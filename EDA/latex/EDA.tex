\documentclass{article}
\usepackage[spanish]{babel}
\usepackage[utf8]{inputenc}
\usepackage[nonatbib]{../../template}

\usepackage[utf8]{inputenc} % allow utf-8 input
\usepackage[T1]{fontenc}    % use 8-bit T1 fonts
\usepackage{hyperref}       % hyperlinks
\usepackage{url}            % simple URL typesetting
\usepackage{booktabs}       % professional-quality tables
\usepackage{amsfonts}       % blackboard math symbols
\usepackage{nicefrac}       % compact symbols for 1/2, etc.
\usepackage{microtype}      % microtypography
\usepackage{xcolor}         % colors
\usepackage{graphicx}
\usepackage{float}
\usepackage[backend=biber,sorting=ynt,style=apa]{biblatex}

\addbibresource{bibliography.bib}

\graphicspath{ {../imagenes/} }

\title{Entrega 2: Metodología y EDA}

\author{%
  José Saint Germain\\
  \texttt{josesg998@gmail.com} \\
}

\begin{document}

\maketitle

\section{Introducción}
El objetivo de esta entrega es realizar una breve descripción de las metodologías 
que se utilizarán durante el trabajo final de especialización, así como realizar 
un análisis exploratorio de los datos (EDA), para comprender mejor la estructura 
delos datos que se trabajarán.

\section{Metodología}
Como lo que buscamos realizar es experimentar con diferentes datos el mismo 
trabajo realizado por el FMI (\cite{Ceb24}), vamos a replicar las mismas 
técnicas de optimización de hiperparámetros, así como los mismos algoritmos
 de entrenamiento y de intepretación de resultados.

Los algoritmos que se utilizarán serán Random Forest (\cite{Bre01}) y XGBoost
(\cite{Che16}). Para ajustar los hiperparámetros se utilizará la optimización
bayesiana junto al método de block-time-series cross-validation. Por último, 
la métrica a optimizar y que se utilizará para comparar predicciones será el
area bajo la curva (AUC). Adicionalmente, se realizará un análisis exploratorio
de datos de manera introductoria al trabajo y se buscará utilizar valores Shapley 
para analizar los resultados de cada algoritmo.


\section{Análisis Exploratorio de Datos}

Como descripción general de la base de datos de VDEM (\cite{Cop24}), podemos mencionar que cuenta 
con 27734 filas y 4607 columnas. Como es una base de datos de panel, se tiene
información de 202 países durante 235 años. Para comprender la estructura de 
la información, es importante destacar que la base original cuenta con información 
brindada por distintos expertos para cada país en cada año. Para poder procesar y 
obtener la base final, se agrega la información de diferentes maneras. Es por este 
motivo que, además de la información identificactoria de cada país (la cual se 
repite en cada año), la mayoría de las variables sustantivas cuentan con diferentes 
versiones, por cada tipo de variable de agregación generada. Por ejemplo, una 
variable puede contar con su versión princial, la cual es un promedio reescalado 
del 1 al 5, sumado a una versión con la media simple (con sufijo \_mean); una 
versión con el valor máximo y mínimo expresado por un experto (\_codehigh y 
\ codelow, respectivamente); y una versión con el desvío estándar (\_sd), en 
caso de buscar conocer el grado de 'acuerdo' entre los expertos respecto a
la situación del país.

A la base original obtenida desde la librería de VDEM, se le realizaron los
siguientes filtros: en primer lugar, se removieron todas las variables que
no sean las principales, es decir, que no cuenten con sufijo. De esa manera,
se busca reducir el tamaño de la base y así poder agregar nuevas columnas
mediante ingeniería de atributos. En segundo lugar, se filtraron los años 
superiores a 1950, para adecuarnos al periodo utilizado en el artiçulo del FMI.
De esa manera, la base filtrada cuenta con 12208 filas y 1460 columnas.

\subsection{Ánalisis de nulos}
Debido a la alta cantidad de variables, no es posible realizar un análisis
pormenorizado de la presencia de nulos en cada una. Por ese motivo, se decidió
visualizar la misma mediante los agrupadores de variables con las que cuenta el
codebook de VDEM. El mismo, discrimina las variables a partir de sus temas en común.
De esa manera, en la figura 1 la cantidad de nulos por categoría de variable
expresado en un mapa de calor, donde cada fila es una variable individual y las
columnas los diferentes años del panel.

\begin{figure}[H]
  \centering  
  \includegraphics[width=1\textwidth]{1_nas.png}
  \caption{Conteo de nulos por año y agrupador de variables}
\end{figure}

De este gráfico podemos aprehender ciertos patrones sobre la presencia de nulos
en algunos grupos de variables: En primer lugar, observamos variables que,
anteriormente a un año puntual, no cuentan con información. En este ejemplo caen
las variables sobre governanza otorgadas por el banco mundial (e7), las preguntas
pertenecientes a la encuesta de sociedad digital (wsmcio), variables referentes a
la libertad en medios digitales (wsmdmf), las referntes a la polarización en medios
online (wsmomp) y las referentes a clivajes sociales (wsmsc).

En segundo lugar, figuran casos contrarios, en donde a partir de determinado año
la cantidad de datos faltantes salta a la totalidad de los casos. En este grupo
figuran las variables asociadas a instituciones y eventos políticos (e13), cuya 
fuente es un artículo de Przeworski de 2013; las variables cuya fuente es la base
de datos polity V (e14); las variables sobre educación (aumentan los nulos en algunas 
variables) (eb1); las variables sobre recursos naturales (eb5), cuya fuente tiene
datos hasta 2006; las variables sobre infraestructura (eb6); y las relacionadas a 
conflictos (eb8). En general, esta discontinuidad sucede debido a que la información
de estas variables provienen de fuentes externas no gestionadas por VDEM, las cuales
finalizaron su serie en un año puntual.

\subsection{Análisis de variable objetivo}

\begin{figure}[H]
  \centering  
  \includegraphics[width=1\textwidth]{2_golpes.png}
  \caption{Conteo de golpes de estado en el mundo}
\end{figure}


\section{Preprocesamiento de los datos}

\textit{Cargar el dataset con los datos para cada sujeto y los nombres y coordenadas 
de las regiones cerebrales a las que se les registró la actividad. Reportar cuántos sujetos y cuántos estados de sueño se observan en el conjunto de
datos.}

\printbibliography

\end{document}