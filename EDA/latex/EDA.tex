\documentclass{article}
\usepackage[spanish]{babel}
\usepackage[utf8]{inputenc}
\usepackage{../../template}



\usepackage[utf8]{inputenc} % allow utf-8 input
\usepackage[T1]{fontenc}    % use 8-bit T1 fonts
\usepackage{hyperref}       % hyperlinks
\usepackage{url}            % simple URL typesetting
\usepackage{booktabs}       % professional-quality tables
\usepackage{amsfonts}       % blackboard math symbols
\usepackage{nicefrac}       % compact symbols for 1/2, etc.
\usepackage{microtype}      % microtypography
\usepackage{xcolor}         % colors
\usepackage{graphicx}
\usepackage{float}


\graphicspath{ {imagenes/} }

\title{Pre TP2: Data Mining en Ciencia y Tecnología}

\author{%
  José Saint Germain\\
  \texttt{josesg998@gmail.com} \\
}

\begin{document}

\maketitle

\section{Introducción}

El análisis de la topología de grafos (es decir, redes) es un área de investigación que
atañe a diferentes campos de estudio. Para ejemplificar el uso de grafos nos enfocaremos en
el los datos obtenidos en el trabajo de Tagliazucchi y colaboradores (2013) que busca 
relacionar cambios en la modularidad de las redes construidas a partir de la señal de resonancia
magnética funcional (fMRI) con los distintos estadíos del sueño.
\section{Objetivos}
Familiarizarse con la generación de grafos que representen un conjunto de datos. Visua-
lizar, manipular y comparar distintos grafos. Calcular parámetros básicos de un grafo, y
compararlos con modelos de redes random, small world y scale-free.

\section{Estructura de los Datos}
En la carpeta DataSujetos se encuentran los archivos separados por cada sujeto y estadio. 
Para cada sujeto y estadío de sueño encontraremos una matriz de correlaciones de tamaño
116x116 con las correlaciones entre las señales BOLD de 116 regiones cerebrales.

Además se incluyen los nombres y coordenadas de las 116 regiones en un archivo apar-
te: aalextendedwithCoords.csv. Estas regiones están definidas a partir del atlas Automatic
Anatomical Labeling (AAL). Ejemplos de los procedimiento para comenzar el análisis pueden 
encontrarse en \href{https://colab.research.google.com/drive/1xU8p_YSeSxPAODgiyJwJAVuuDC-jrTwP#scrollTo=VG4joS9_OZCA#offline=true&sandboxMode=true}{este colab}.

\section{Preprocesamiento de los datos}

\textit{Cargar el dataset con los datos para cada sujeto y los nombres y coordenadas 
de las regiones cerebrales a las que se les registró la actividad. Reportar cuántos sujetos y cuántos estados de sueño se observan en el conjunto de
datos.}

% \begin{table}[h]
%   \centering
%   \caption{Conteo de estados de sueño en el conjunto de datos.}
%   \label{tab:estados_sueno}
%   \begin{tabular}{lr}
%     \toprule
%     estado & cantidad de sujetos \\
%     \midrule
%     N1 & 18 \\
%     N2 & 18 \\
%     N3 & 18 \\
%     W & 18 \\
%     \bottomrule
%   \end{tabular}
% \end{table}

% Como se observa en el cuadro 1, el dataset cuenta con 18 sujetos, teniendo cada uno lo
% cuatro diferentes estados de sueño.

% \section{Manipulación de datos}

% \textit{5.1 Graficar la matriz de correlaciones entre regiones (es decir, la "matriz de adyacencia
% pesada") para el sujeto 2 de la condición despierto ("Wake"). Transformar dicha matriz de adyacencia pesada a una matriz de adyancia binaria $A_{i,j}$
% que represente una una densidad de enlaces igual a 0.08. ¿Cuál es el valor de umbral de correlación entre pares de regiones que tuvo que utilizar?}

% Como se observa en la figura 1, se grafica en la imagen de la izquierda la matriz de
% correlaciones entre regiones. A su vez, en la imagen de la derecha, se grafica la misma
% matriz representando una densidad de enlaces igual a 0.08. Para ello, se utilizó un umbral
% de 0.75, el cual se obtiene con la función desarrollada por el profesor, utilizando como
% hiperparámetro la densidad de enlaces deseada.

% \begin{figure}[H]
%   \centering  
%   \includegraphics[width=.8\textwidth]{1_Sujeto2W.png}
%   \caption{Matriz de adyacencia pesada y binarizada del sujeto 2 despierto}
% \end{figure}


% \textit{5.2 Utilizando $A_{i,j}$ , obtener el grafo resultante G ¿Es G un grafo conectado? ¿Se puede calcular la distancia media entre pares de nodos
% d del grafo G? ¿Si no se puede, qué medida equivalente calcularías?}

% El grafo que se obtiene de $A_{i,j}$ no está conectado, por lo que no se puede calcular
% la distancia media entre pares de nodos. Alternativamente, se puede calcular la distancia media de su
% componente gigante (es decir, el componente conectado más grande dentro del grafo G),
% el cual es 3.85. Otra opción es calcular la eficiencia global del grafo el cual se calcula en el ejercicio
%  siguiente.

% \textit{5.3 Calcular d para cada componente conectado de G. Calcular la eficiencia global ef f del
% grafo G.}

% La distancia d de los dos componentes conectados más grandes de G dan 3.85 y 1.28. Después,
% el resto de componentes conectados tienen un valor de d igual a 0. Por último, la eficiencia
% global del grafo G es 0.2446.

% \textit{5.4 Obtener la lista de enlaces del grafo G. Calcular el grado promedio < k >, el nodo con grado máximo kmax, el coeficiente de 
% clustering promedio}

% \begin{table}[H]
%   \begin{minipage}{0.5\linewidth}
%   \centering
%   \caption{Enlaces aleatorios del grafo G}
%   \begin{tabular}{rr}
%     \toprule
%     Nodo 1 & Nodo 2 \\
%     \midrule
%     23 & 92 \\
%     10 & 44 \\
%     11 & 62 \\
%     28 & 99 \\
%     4 & 53 \\
%     60 & 66 \\
%     0 & 28 \\
%     95 & 97 \\
%     \bottomrule
%   \end{tabular}
% \end{minipage}%
% \begin{minipage}{0.5\linewidth}
%   \centering
%   \caption{Medidas del grafo G}
%   \begin{tabular}{lr}
%   \toprule
%   Medida & Valor \\
%   \midrule
%   Grado promedio (K) & 9.207 \\
%   Nodo con grado máximo (kmax) & 30 \\
%   Coeficiente de clustering promedio (C) & 0.527 \\
%   Eficiencia global  & 0.245 \\
%   \bottomrule
%   \end{tabular}
% \end{minipage}
% \end{table}

% Ya que la lista de enlaces del grafo es extensa. Se muestra una lista aleatoria de 8 enlaces 
% del grafo G. También se muestran los valores de las medidas del grafo G en el cuadro 3.

% \textit{5.6 Visualizar el grafo, ubicando los nodos en sus coordenadas cerebrales y coloreando cada
% nodo de acuerdo a su coeficiente de clustering Ci. Graficar la distribución de grado del grafo, elijiendo un número de bins apropriado}


% \begin{figure}[H]
%   \centering  
%   \includegraphics[width=.9\textwidth]{grafo_hist.png}
%   \caption{Grafo G}
% \end{figure}


% \textit{5.7 Vamos a comparar el grafo G con prototipos de redes poissonianas (random), small-World
%  y scale-free, usando los algoritmos de Erdos-Renyi, Watts-Strogatz y Barabasi-Albert, respectivamente.
%  Para ello, elegir (y reportar) los parámetros utilizados para cada algoritmo, 
% buscando siempre que los grafos simulados de dichos prototipos sean comparables al grafo de datos G 
% (en términos de número de nodos y números de enlaces). Visualizar un ejemplo de grafo para cada uno de 
% estos prototipos de redes. Discutir diferencias.}

% A continuación, se enlistan los distintos parámetros utilizados para generar cada tipo
% de grafo:

% \begin{table}[h]
%   \caption{Parámetros de las redes aleatorias}
%   \begin{tabular}{lrrllll}
%   \toprule
%   Red & Nodos & Enlaces & Parámetro 2  &  & Parámetro 3 &  \\
%       &       &         & (scale free y small world) &  & (small world) &  \\
%   \midrule
%   Poisson & 116 & 534 & - & - & - & - \\
%   Small world & 116 & 464 & Vecinos conectados  & 9 & Prob. de  & 0.02 \\
%               &     &     & a cada nodo         &    & re-conexión   &\\
%   Libre de escala & 116 & 555 & Q de enlaces nuevos & 5 & - & - \\
%                   &     &     &  por nodo nuevo     &&&\\
%   \bottomrule
%   \end{tabular}
%   \end{table}
  
% Nota: para la red poissoniana no se aclaran parámetros extra porque los que se
% utilizaron fueron al cantidad de nodos y la de enlaces.

% Adicionalmente, se grafican los mismos con una disposición basada en 
% el algoritmo de Fruchterman Reingold:

% \begin{figure}[H]
%   \centering  
%   \includegraphics[width=.7\textwidth]{4_Sujeto2W.png}
%   \caption{Visualización de un grafo de cada prototipo}
% \end{figure}

% Con respecto a la red poissoniana, no hay mucho para destacar, 
% dado su comopnente aleatorio. La red Small-world llama la atención por
% tener muchos nodos encadenados, con varios grupos de nodos sin esa
% forma. Por último, la red libre de escala a simple vista tienen una forma
% muy similar a la poissoniana.

% \textit{5. Generar 1000 instancias de grafos para cada uno de dichos prototipos (poissonianas,
% small-World y scale-free). Para el conjunto de 1000 instancias de cada prototipo, calcular el histograma de coeficientes de < k >, kmax, C, y ef f . Comparar con los valores
% de coeficientes que obtuvimos para el grafo de datos G.}

% \begin{figure}[H]
%   \centering  
%   \includegraphics[width=1\textwidth]{5_Sujeto2W.png}
%   \caption{Comparación de métricas de grafo G con el resto}
% \end{figure}

% %TODO: comparar con los valores de coeficientes que obtuvimos para el grafo de datos G
% Con respecto al grado promedio del grafo (K), se observa que el grafo G tienen el mismo valor que
% la red poissoniana. La red Small-world tiene un menor grado promedio, lo cual hace sentido
% con la imagen observada anteriormente, en donde una gran cantidad de nodos apenas se conectan
% con otros dos. Por último, la red libre de escala tiene un grado promedio mayor que el grafo G;
% indicando la presencia de hubs con alto grado.

% En cuanto al grado máximo (kmax), se observa que el grafo G tiene un valor mayor que la Poisson.
% También es mayor que el Small-world, cuyos valores están más concentrados que en la Poisson.
% Por último, el grado máximo de G aparece entre los valores más chicos del grado máximo de la 
% red libre de escala, acompañando la misma idea planteada para el grado promedio.

% Tomando el coeficiente de clustering medio (C), el grafo G tiene un valor mayor que la red Poisson,
% indicando que los vecinos de cada nodo están mucho más conectados. Por otro lado, el Small-world
% tiene un valor más alto. Se puede entender que al tener una probalidad de reconexión baja (0.02)
% la mayoría de los nodos están conectados con su primer y segundo vecino, los cuales estarían
% conectados entre sí. Después, el libre de escalas tiene valores menores al del G, indicando
% la presencia de hubs, que conectan nodos que no están conectados entre sí.

% Finalmente, la efiencia global (eff) del grafo G es mucho menor que las tres redes. Esto evidencia
% que la distancia promedio de los nodos de G es mucho mayor. De todas formas, el valor de G se acerca
% a los valores mínimos de la Small-world, lo cual puede entenderse si tiene una probabildiad de reconexión
% baja ya que es muy probable que tengas pocos atajos (enlace "recableado") para llegar de un nodo a otro.

\end{document}